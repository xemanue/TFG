\begin{center}
    \large\bfseries Aplicación de control para testeador de faros de vehículos
\end{center}

\begin{center}
    Jose Manuel García Cazorla\\
\end{center}

\noindent{\textbf{Palabras clave}: firmware, microcontrolador, Arduino, PWM, aplicación, frontend, C, Python, QT}\\

\vspace{0.7cm}
\noindent{\textbf{Resumen}}\\

La industria del automóvil es un sector en constante desarrollo, siempre a la vanguardia de los avances tecnológicos para producir vehículos cada vez más sofisticados. Tratándose del medio de transporte más usado en la actualidad, cada pieza es sumamente importante: por muy prescindible que parezca, puede llegar a salvar vidas.

Uno de los componentes que más se pueden dan por sentado son los faros. Aunque se usen solo una fracción del tiempo total que pasamos al volante, sin ellos no podríamos conducir de noche, o en situaciones en las que la visibilidad no sea óptima.

En este trabajo se presenta el análisis, diseño y desarrollo del firmware de PWM Box, un dispositivo que genera señales PWM configurables por el usuario con el objetivo de probar faros de vehículos. Asimismo, se implementa una interfaz gráfica que permite gestionar distintos perfiles y parámetros de configuración del dispositivo desde un ordenador.

\cleardoublepage

\thispagestyle{empty}

\begin{center}
    Control app for a vehicle headlight tester
\end{center}

\begin{center}
    Jose Manuel García Cazorla\\
\end{center}

\noindent{\textbf{Keywords}: firmware, microcontroller, Arduino, PWM, app, frontend, C, Python, QT}\\

\vspace{0.7cm}
\noindent{\textbf{Abstract}}\\

The car industry is an ever-developing sector, always on the cutting edge of technological advances to produce increasingly sophisticated vehicles. Currently being the most used means of transportation, each of its components is extremely important: no matter how expendable it seems, it may even help to save lives.

One of the most easily overlooked components are headlights. Even though we only use them for a fraction of the time we spend on our cars, we wouldn't be able to drive at night or in low-visibility situations without them.

This project showcases the analysis, design and development of the firmware for the PWM Box, a device that generates user-configurable PWM signals intended for testing vehicle headlights. Additionally, a graphic interface is implemented, which allows to manage the device's profiles and other configuration parameters from a computer.

\chapter*{}
\thispagestyle{empty}

\noindent\rule[-1ex]{\textwidth}{2px}\\[4.5ex]

Yo, \textbf{Jose Manuel García Cazorla}, alumno del Grado en Ingeniería Informática de la \textbf{Escuela Técnica Superior de Ingernierías Informática y de Telecomunicación de la Universidad de Granada}, con DNI 15434710P, autorizo la ubicación de la siguiente copia de mi Trabajo de Fin de Grado en la biblioteca del centro para que pueda ser consultada por las personas que lo deseen.

\vspace{6cm}

\noindent Fdo.: Jose Manuel García Cazorla

\vspace{2cm}

\begin{flushright}
    Granada, a 1 de Septiembre de 2025.
\end{flushright}

\chapter*{}
\thispagestyle{empty}

\noindent\rule[-1ex]{\textwidth}{2px}\\[4.5ex]

D. \textbf{Andrés María Roldán Aranda}, Profesor del área de TODO del Departamento de Electrónica y Tecnología de Computadores de la Universidad de Granada.

\vspace{0.5cm}

\textbf{Informan:}
Que el presente trabajo, titulado \textit{\textbf{Aplicación de control para testeador de faros de vehículos}}, ha sido realizado bajo su supervisión por \textbf{Jose Manuel García Cazorla}, y autorizamos la defensa de dicho trabajo ante el tribunal que corresponda.

\vspace{0.5cm}

Y para que conste, expiden y firman el presente informe en Granada a 1 de Septiembre de 2025.

\vspace{1cm}

\textbf{El director:}

\vspace{5cm}

\noindent\textbf{Andés María Roldán Aranda}

\chapter*{Agradecimientos}
\thispagestyle{empty}

\vspace{1cm}

% Doy las gracias a Andrés, mi tutor de este Trabajo de Fin de Grado, por guiarme en esta dura etapa.

% Agradecer también a mi familia y a mi pareja por su incondicional apoyo y ánimos a lo largo de este proceso. Sin ellos, me habría sido imposible alcanzar este logro.
