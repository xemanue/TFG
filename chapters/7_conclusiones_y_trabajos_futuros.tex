\chapter{Conclusiones y trabajos futuros}

\section{Conclusiones}

Este proyecto perseguía el objetivo de implementar nuevas funcionalidades en un dispositivo testeador de faros de vehículos, así como desarrollar una interfaz gráfica que lo complemente.

Para ello, se ha implementado en el firmware del PWM Box la capacidad de almacenar perfiles de configuración en la memoria EEPROM del microcontrolador. También se ha desarrollado un nuevo menú, que permite comprobar el funcionamiento de faros que funcionen a una frecuencia menor de la que es capaz de generar el dispositivo usando señales PWM. Adicionalmente, se han corregido varios errores existentes que afectaban a la experiencia del usuario.

En cuanto a la interfaz, se ha usado el \textit{framework} Qt en combinación con PySide6 para diseñar e implementar una aplicación multiplataforma que permita almecenar y enviar distintos perfiles al dispositivo. Además, se ha incluido la capacidad de exportar dichos perfiles en formato JSON, de forma que puedan ser importados de nuevo cuando se necesite.

Todo esto se ha llevado a cabo teniendo en cuenta los requisitos propuestos en las secciones \ref{sec:fw_diseño} y \ref{sec:ui_diseño} respectivamente. Por ello, se puede considerar que los objetivos marcados se han cumplido eficazmente.

En lo personal, estoy bastante satisfecho con el resultado final del proyecto. Me ha permitido poner en práctica muchos de los conocimientos adquiridos a lo largo del Grado, saliendo en ocasiones de mi zona de comfort para obtener otros nuevos.

Si tuviera que mencionar un aspecto en el que considero que podría haberlo hecho mejor, sería en la gestión del tiempo. En muchas ocasiones a lo largo del desarrollo, he dejado que mi opinión personal sobre qué estaba bien y qué estaba ``menos bien'' dictara la parte que merecía más trabajo. Esto me ha hecho pasar demasiado tiempo intentando perfeccionar algunas secciones que no lo necesitaban realmente. En el futuro trataré de evitar guiarme en exceso por mi juicio, y ceñirme mejor a la planificación inicial.

Esto no quita que, a la vez, me sienta orgulloso de haber sido capaz de compaginar este proyecto con un puesto a tiempo completo de 43 horas semanales, y aun así haber obtenido un resultado que considero positivo.

\section{Trabajos futuros}

Aunque los cambios realizados hayan sido suficientes para completar el objetivo actual, aún cabe la posibilidad de mejorar el producto.

En primer lugar, el número de ranuras que puede almacenar actualmente el dispositivo es limitado, permitiendo un máximo de sólo 10 perfiles. Esto no se debe a un límite en el tamaño de la memoria EEPROM, sino a que la forma en la que están declaradas sus variables para permitir un mayor rendimiento ocupa mucho espacio en la RAM. Con una mejor gestión de la misma a lo largo de todo el firmware, se podrían guardar un mayor número de ranuras.

Esta ampliación de la memoria nos lleva a una siguiente mejora posible: la velocidad de transmisión. Actualmente, enviar el máximo número de ranuras posibles conlleva una espera de alrededor de dos minutos. Esto se debe a las limitaciones de la comunicación por el puerto serie, que requiere una espera entre cada envío de datos para asegurar que el dispositivo los reciba correctamente. Si se amplía el número de ranuras, también se incrementa el tiempo de espera al trasmitir este número de ranuras al PWM Box. Para solucionarlo, sería necesario plantear el uso de otro tipo de comunicación, o explorar la posibilidad de usar más de una UART al mismo tiempo.

Por último, aunque la interfaz gráfica se pretende usar desde un ordenador, siempre cabe la posibilidad de hacerla compatible con un mayor número de dispositivos. Para ello, se podría considerar una implementación usando otros \textit{frameworks} como Flutter o Electron.
